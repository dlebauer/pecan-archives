\documentclass[12pt]{article}

%% Math
\usepackage{amsmath}  % for math formulas
\usepackage{amssymb}  % for math symbols
\usepackage{bm}
\renewcommand{\vec}[1]{\bm{#1}} %% define vector notation

%% Tables
\usepackage{rotating} % enable sidewaystable
\usepackage{longtable} % used for long tables of data in appendix
\usepackage{booktabs}  % also for long tables of data in appendix
\usepackage{subfig}
%% Figures
\usepackage{graphicx}
%% all fig. legends on a separate page
%% http://tex.stackexchange.com/q/30477/1783
\usepackage{caption}
\usepackage{letltxmacro}% http://ctan.org/pkg/letltxmacro
\captionsetup{labelsep=none}
\DeclareCaptionTextFormat{none}{} \captionsetup{labelsep=none,textformat=none}

%% Text layout
\usepackage{setspace} 
\usepackage{geometry}
\geometry{verbose,a4paper,tmargin=2.4cm,bmargin=2.4cm,lmargin=2.4cm,rmargin=2.4cm}
\usepackage{lineno}
\linenumbers
\usepackage{Sweave}% use of Sweave with R code
\usepackage{natbib}
\bibliographystyle{ecology}

\title{Appendix}
\date{} % Leave date blank
\begin{document}

\begin{flushleft}
\begin{spacing}{1.9}


\section{Switchgrass Data}
\end{spacing}
\begin{spacing}{1}
\label{app:data}
All data used in the present analysis, along with site, citation, and treatment information, are available in the BETY database. Each data point is identified by a unique trait\_id: id is the primary key in the traits table, and trait\_id is the foregin key in auxillary tables (Appendix B).

\subsection*{Present study data}
 
\subsubsection*{Vcmax and SLA}
V$_{\text{cmax}}$ and SLA measurements were made on four year old switchgrass (\emph{P.~virgatum}) stands were grown in an agricultural study site in Savoy, IL (40$^o$10'20"N, 88$^o$11'40"W, 228 m above sea level).
Gas exchanges were measured on leaves with a portable infrared gas analyzer (LI-COR 6400LCF; Li-COR, Lincoln, NE). 
During measurements, leaves were exposed to a CO$_2$ concentration of 370 $\mu$mol mol$^{-1}$, temperature at $25^o$C, vapor pressure deficit (VPD) at the leaf surface 1.5 kPa and airflow through the chamber 250 $\mu$mol s$^{-1}$. For the CO$_2$ response (A-Ci) curves, leaves were acclimated for 30-60 minutes before adjusting the CO$_2$ concentrations. 
Thereafter, CO$_2$ concentration was decreased in 5 steps (400, 300, 200, 100 and 50 ppm CO$_2$) and then increased in 3 steps (400, 600 and 800 μmol mol-1 CO$_2$). A-Ci curves were fitted to a coupled photosynthesis-stomatal conductance model \citep{collatz1992cps}. 
The rate saturated region of the A-Ci curves were used to estimate maximum Rubisco activity (Vcmax) \citep{miguez2009smp}. 

SLA was computed as the ratio of leaf area to mass. Ten 0.5 cm$^2$ leaf punches from 4 different plants were taken and oven-dried at 65 $^o$C for two weeks and then weighed.

\subsubsection*{Stomatal Slope data}

Stomatal slope was estimated using measurements of four leaves from each of five field-grown energy crop species during the 2010 growing season. 
The five species included two C4 grasses: Miscanthus (\textit{Miscanthus~x~giganteus}) and Switchgrass (\textit{P~virgatum}) planted in 2008 and three deciduous tree species: Red Maple (\textit{Acer~rubrum}), Eastern Cottonwood (\textit{Populus~deltoides}, and Sherburne Willow \textit{Salix~x~Sherburne}) planted in 2010 as 2 year old saplings. 
 All plants were grown at the Energy Biosciences Institute Energy Farm (40$^o$10'N, 88$^o$03"W).

Photosynthesis (A), stomatal conductance (gs), intercellular [CO$_2$] (ci), and humidity deficit at the leaf surface (Ds) were obtained via open gas exchange systems with 2 cm$^2$ leaf chambers housing infrared gas analyzers to measure fluxes of both CO$_2$ and water (LI-6400; LI-COR Inc., Lincoln, NE, USA). 
Data were collected following a simplified version of the protocol described by \citet{leakey2006ltg} in which photosynthetic photon flux density was maintained at 1500 $\mu$mol m$^{-2}$ s$^{-1}$, leaf temperature was $25 \pm 3^o$C and the vapor pressure deficit from leaf to air was < 2 kPa while [CO$_2$] entering the chamber was varied stepwise (400, 250, 350, 450, 650, 850, 1200, 1500 ppm). A minimum of 20 minutes was allowed for A and gs to completely stabilize before data were collected at each [CO$_2$]. 
 For each individual leaf, linear least squares regression was used to estimate the stomatal slope based on the \citet{ball1987mps} model of stomatal conductance (not used in present study but provided as data in appendix), and then separately for the \citet{leuning1995cac} model of stomatal conductance. A common value of $\Gamma=40 \mu$P Pa$^{-1}$, and D$_0=1500$ Pa was used in accordance with \citet{leuning1995cac}.

{\small
\begin{longtable}{rrrlr}
  \toprule
  Mean & n & SE & BETY trait\_id \\
  \midrule \endhead
  Leuning slope parameter & & & & \\*
  \hline
  4.35& 1& 	0.51 & 40909\\
  3.93& 1& 	0.13 & 40910\\
  3.74& 1& 	0.21 & 40911\\
  4.37& 1& 	0.33 & 40912\\
  \hline
  SLA ($g C/ m^2$) & & & & \\*
  \hline
  34.5 &  2 & 12.2  & 2592 \\* 
  28.4 &  2 & 4.7  & 2593 \\* 
  32.1 &  2 & 3.6  & 2597 \\* 
  30.5 &  2 & 5.7  & 2598 \\* 
  \hline
  Vcmax & & &  \\*
  \hline
  18.1 &  2 & 6.2 & 2638 \\* 
  16.3 &  2 & 2.9 & 2639 \\* 
  8.9 &  2 & 4.8  & 2640 \\* 
  8.8 &  2 & 6.97 & 2641 \\* 
  20.8 &  2 & 7.5 & 2642 \\* 
  14.4 &  2 & 5.8 & 2643 \\* 
  16.9 &  2 & 8.4 & 2644 \\* 
  6.2 &  2 & 2.1  & 2645 \\
  \hline
  \bottomrule
  \caption{ Unpublished switchgrass trait data collected for the present analysis.}
\end{longtable}
}
\newpage
\subsection*{Previously published data}
\begin{longtable}{rrrlr}
  \toprule
  Mean & n & SE & citation & BETY trait\_id \\
  \midrule \endhead
  SLA ($g C/ m^2$) & & & & \\*
  \hline
  38.8 &  2 & 1.0 & \citet{knapp1995efd} & 132 \\* 
  40.6 &  2 & 2.2 & \citet{knapp1995efd} & 133 \\* 
  40.8 &  8 &  & \citet{byrd2000pcs} & 281 \\* 
  39.6 &  8 &  & \citet{byrd2000pcs} & 282 \\* 
  49.5 &  8 &  & \citet{byrd2000pcs} & 283 \\* 
  51.7 &  4 &  & \citet{byrd2000pcs} & 285 \\* 
  53.3 &  4 &  & \citet{byrd2000pcs} & 286 \\* 
  46.4 &  4 &  & \citet{byrd2000pcs} & 287 \\* 
  54.2 &  4 &  & \citet{byrd2000pcs} & 288 \\* 
  58.0 &  4 &  & \citet{byrd2000pcs} & 289 \\* 
  52.8 &  4 &  & \citet{byrd2000pcs} & 290 \\* 
  45.2 &  4 &  & \citet{trocsanyi2009ycc} & 8478 \\* 
  37.9 &  4 &  & \citet{trocsanyi2009ycc} & 8482 \\* 
  38.5 &  4 &  & \citet{trocsanyi2009ycc}  & 8487 \\
  \hline
  \hline
  fine root:leaf & & & &  \\*
  \hline
  0.59 &  4 &  & \citet{kiniry1999rue} & 22092 \\* 
  2.73 &  4 &  & \citet{kiniry1999rue} & 22093 \\* 
  0.43 &  4 &  & \citet{kiniry1999rue} & 22094 \\* 
  1.5 &  4 &  & \citet{kiniry1999rue}  & 22095 \\* 
  1.81 &  2 & 0.27 & \citet{tjoelker2005llr} & 25670 \\* 
  0.74 &  2 & 0.30 & \citet{tjoelker2005llr} & 25675 \\
  \hline

  \newpage
  \hline
  leaf width (mm) & & & & \\*
  \hline
  10.2 & 2 & 0.27 & \citet{knapp1995efd} & 136 \\* 
  5.9 &  2 & 0.23 & \citet{knapp1995efd} & 137 \\* 
  5.0 &  2 & 0.18 & \citet{redfearn1997cam} & 332 \\* 
  4.9 &  2 & 0.18 & \citet{redfearn1997cam} & 333 \\* 
  6.4 &  2 & 0.18 & \citet{redfearn1997cam} & 334 \\* 
  6.3 &  2 & 0.18 & \citet{redfearn1997cam} & 335 \\* 
  6.2 &  2 & 0.18 & \citet{redfearn1997cam} & 336 \\* 
  7.2 &  2 & 0.18 & \citet{redfearn1997cam} & 337 \\* 
  5.2 &  2 & 0.18 & \citet{redfearn1997cam} & 338 \\* 
  4.6 &  2 & 0.18 & \citet{redfearn1997cam} & 339 \\* 
  6.2 &  2 & 0.18 & \citet{redfearn1997cam} & 340 \\* 
  5.8 &  2 & 0.18 & \citet{redfearn1997cam} & 341 \\* 
  4.6 &  2 & 0.18 & \citet{redfearn1997cam} & 342 \\* 
  6.8 &  2 & 0.18 & \citet{redfearn1997cam} & 343 \\* 
  6.8 &  2 & 0.18 & \citet{redfearn1997cam} & 344 \\* 
  6.6 &  2 & 0.18 & \citet{redfearn1997cam} & 345 \\* 
  7.9 &  2 & 0.18 & \citet{redfearn1997cam} & 346 \\* 
  7.4 &  2 & 0.18 & \citet{redfearn1997cam} & 347 \\* 
  6.7 &  2 & 0.18 & \citet{redfearn1997cam} & 348 \\* 
  7.0 &  2 & 0.18 & \citet{redfearn1997cam} & 349 \\* 
  4.8 &  2 & 0.18 & \citet{redfearn1997cam} & 386 \\* 
  4.8 &  2 & 0.18 & \citet{redfearn1997cam} & 387 \\* 
  6.2 &  2 & 0.18 & \citet{redfearn1997cam} & 388 \\* 
  5.8 &  2 & 0.18 & \citet{redfearn1997cam} & 389 \\* 
  4.7 &  2 & 0.18 & \citet{redfearn1997cam} & 390 \\* 
  7.7 &  2 & 0.18 & \citet{redfearn1997cam} & 391 \\* 
  4.6 &  2 & 0.18 & \citet{redfearn1997cam} & 392 \\* 
  5.6 &  2 & 0.18 & \citet{redfearn1997cam} & 393 \\* 
  7.3 &  2 & 0.18 & \citet{redfearn1997cam} & 394 \\* 
  6.6 &  2 & 0.18 & \citet{redfearn1997cam} & 395 \\* 
  7.3 &  2 & 0.18 & \citet{redfearn1997cam} & 396 \\* 
  7.0 &  2 & 0.18 & \citet{redfearn1997cam} & 397 \\* 
  5.1 &  2 & 0.18 & \citet{redfearn1997cam} & 398 \\* 
  4.7 &  2 & 0.18 & \citet{redfearn1997cam} & 399 \\* 
  5.8 &  2 & 0.18 & \citet{redfearn1997cam} & 400 \\* 
  6.4 &  2 & 0.18 & \citet{redfearn1997cam} & 401 \\* 
  5.0 &  2 & 0.18 & \citet{redfearn1997cam} & 402 \\* 
  7.0 &  2 & 0.18 & \citet{redfearn1997cam} & 403 \\* 
  7.6 &  2 & 1.70 & \citet{oyarzabal2008tdb} & 453 \\ 
  \hline
  \caption{ Previously published switchgrass trait data used in meta-analysis}
\end{longtable}
\newpage

\section{BETYdb}
\label{app:betydb}

 The Biofuel Ecophysiological Traits and Yields database (BETYdb, http://ebi-forecast.igb.uiuc.edu) structure (Figure \ref{fig:betydbschema}). Comprehensive documentation of database structure and web-based data entry is available at the website.

\begin{figure}[ht!]    
 \includegraphics[width=3in]{betydbschema}
 \caption{ Database schema for the traits database.}
 \label{fig:betydbschema}
\end{figure}

\clearpage
\section{Transformations}

% \subsection{Unit conversions}

% The following non-standard conversions were used to convert data from reported units to the standardized units used in BETYdb. 
% \begin{table}[ht]
%   \caption{Unit conversions}
%   \label{tab:traitconversion}
%   \begin{tabular}{lllp{2in}}
%     \hline
%     from ($X$) & to ($Y$) & conversion                      & notes\\ \hline
%      $X_1=$biomass, $X_2=$production & root turnover rate & $Y = X_2/X_1$&  \citet{gill2000gpr}\\
%     US ton/acre &Mg/ha &$Y = X * 2.24$ & \\
%     \% roots &root:shoot (q)& $Y=\frac{X}{1-X}$& $\% \text{roots} = \frac{\text{root biomass}}{\text{total biomass}}$ \\
%     mm s$^{-2}$&mmol m$^{-3}$ s$^{-1}$ &$Y=X/41$ &\citet{korner1988gsc} \\
%     mg CO$_2$ g$^{-1}$ h$^{-1}$ & $\mu$mol kg$^{-1}$ s$^{-1}$& $Y = X\times 6.31$& root respiration rate\\
%     \hline
%   \end{tabular}
% \end{table}


\subsection*{Arrhenius correction}\label{app:arrhenius}
 Parameters for enzyme kinetics ($V_{c_{max,T_m}}$ and root respiration rate) were scaled from the measurement temperature ($T_o$) to a standard temperature ($T_m=298 K (= 25^oC)$) using an Arrhenius correction:
$$V_{c_{max,T_m}}=\frac{V_{c_{max,T_0}}}{e^{3000*(1/(T_o)-1/(T_m))}}$$  

\subsection*{Estimating SE from reported statistics}\label{app:seest}

Often, differences between treatments are reported with P-values, least significant differences (LSD), and other statistics but provide no direct estimate of the variance.
It is reasonable to always assume that the statistics were calculated using the assumption that the data are normally distributed.

\begin{enumerate}
\item given MSE and $n$
 $$SE=\sqrt{MSE/n}$$
\item given $P$, $n$, and treatment means $\bar X_1$ and $\bar X_2$

$$SE=\frac{\bar X_1-\bar X_2}{t_{(1-\frac{P}{2},2n-2)}\sqrt{2/n}}$$

\item given LSD, $\alpha$, $n$, $b$ where $b$ is number of blocks $^{1}$, and $n=b$unless otherwise specified for a randomized complete block design \citep{rosenberg2004map}:

$$SE = \frac{LSD}{t_{(0.975,n)}\sqrt{2bn}}$$

\item given MSD (minimum significant difference) given $n$, $\alpha$, df = $2n-2$  \citep{wang2000asp}

$$SE = \frac{MSD}{t_{(0.975, 2n-2)}\sqrt{2}}$$

\item given a 95\% Confidence Interval (measured from mean to upper or lower confidence limit), $\alpha$, and $n$  \citep{saville2003bsi}
$$SE = \frac{CI}{t_{(\alpha/2,n)}}$$
\item given Tukey's HSD, $n$, where $q$ is the `studentized range statistic', $$SE = \frac{HSD}{q_{(0.975,n)}}$$
\item To solve for $MSE$ given $F$, $df_{\textrm{group}}$, and $SS$ (required when a partial anova table is provided)
The definition $F = MS_g/MS_e$, where $g$ indicates the group, or treatment can be rearranged to solve for the MSE: $MS_e=MS_g/F$
 Then if $MS_x = SS_x/df_x$, we can substitute $SS_g/df_g$ for $MS_g$ in the definition of $F$: $F=\frac{SS_g/df_g}{MS_e}$ and then solve for $MS_e$: $MS_e = \frac{SS_g}{df_g\times F}$.

\end{enumerate}
 In the present study, all required transformations were done prior to entry in the database using these formulas. 
 Subsequently, the PEcAn function \texttt{transformstats} has been developed to automate transformations of SD, MSE, LSD, 95\%CI, HSD, and MSD to conservative estimates of SE.

\subsection*{Calculating precision from SE}

Given variance ($\sigma^2=\frac{1}{N}\sum(i_i-\mu)^2$), sd $\left(\sigma=\sqrt{\sigma^2}=\sqrt{\frac{1}{N}\sum(i_i-\mu)^2}\right)$, and se ($se=\frac{\sigma}{\sqrt{n}}$), calculate precision $\tau$:

$$\sigma=se*\sqrt{n}$$
$$\sigma^2=se^2*n$$
$$\tau=\frac{1}{\sigma^2}=\frac{1}{se^2*n}$$

\section{Derivation of a Gamma prior on $\tau$}

$$\tau \sim G(\frac{n}{2}, \frac{\sum_{i=1}^{n}(\mu-x_i)^2}{2})$$

$$1/\tau_0=\sigma^2=\frac{\sum_{i=1}^{n}(\mu-x_i)^2}{n}$$

$$n/\tau_0=n\sigma^2=\sum_{i=1}^{n}(\mu-x_i)^2$$
  
$$\tau \sim IG(\frac{n}{2}, \frac{n}{2\tau})$$

\newpage
\bibliography{pecan_manuscript}
\end{spacing}
\end{flushleft}
\end{document}